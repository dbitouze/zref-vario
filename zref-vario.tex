% \iffalse meta-comment
%
% File: zref-vario.tex
%
% This file is part of the LaTeX package "zref-vario".
%
% Copyright (C) 2022  Gustavo Barros
%
% It may be distributed and/or modified under the conditions of the
% LaTeX Project Public License (LPPL), either version 1.3c of this
% license or (at your option) any later version.  The latest version
% of this license is in the file:
%
%    https://www.latex-project.org/lppl.txt
%
% and version 1.3 or later is part of all distributions of LaTeX
% version 2005/12/01 or later.
%
%
% This work is "maintained" (as per LPPL maintenance status) by
% Gustavo Barros.
%
% This work consists of the files zref-vario.dtx,
%                                 zref-vario.ins,
%                                 zref-vario.tex,
%                                 zref-vario-code.tex,
%                   and the files generated from them.
%
% The released version of this package is available from CTAN.
%
% -----------------------------------------------------------------------
%
% The development version of the package can be found at
%
%    https://github.com/gusbrs/zref-vario
%
% for those people who are interested.
%
% -----------------------------------------------------------------------
%
% \fi

\documentclass{l3doc}

% The package itself *must* be loaded so that \GetFileInfo can pick up date
% and version data.
\usepackage{zref-vario}

\usepackage[T1]{fontenc}

\usepackage[sc]{mathpazo}
\linespread{1.05}
\usepackage[scale=.88]{tgheros} % sans
\usepackage[varqu,scaled=1.03]{inconsolata} % tt

\usepackage{listings}

\usepackage{microtype}

\hypersetup{hidelinks}

\NewDocumentCommand\opt{m}{\texttt{#1}}

\setlength{\marginparsep}{2\labelsep}

% Temporary workaround for 'l3doc' bug.
% https://github.com/latex3/latex3/issues/1111
% FIXME Remove when fixed upstream.
\NewDocElement[macrolike=false]{Option}{option}

\lstdefinestyle{code}{
  language=[LaTeX]TeX,
  % moretexcs={
  % }
}
\lstdefinestyle{zrefvario}{
  style=code,
  % moretexcs={
  % }
}
\lstset{
  style=zrefvario,
  basicstyle=\ttfamily\small,
  columns=fullflexible,
  keepspaces,
  xleftmargin=\leftmargin,
  xrightmargin=.5\leftmargin,
}
\lstnewenvironment{zvexample}[1][]{%
  \renewcommand{\lstlistingname}{Example}%
  \lstset{#1}%
}{}

\begin{document}

\GetFileInfo{zref-vario.sty}

\title{%
  The \pkg{zref-vario} package%
  \thanks{This file describes \fileversion, released \filedate.}%
  \texorpdfstring{\\{}\medskip{}}{ - }%
  User manual%
  \texorpdfstring{\medskip{}}{}%
}

\author{%
  Gustavo Barros%
  \thanks{\url{https://github.com/gusbrs/zref-vario}}%
}

\date{\filedate}

\maketitle

\begin{center}
  {\bfseries \abstractname\vspace{-.5em}\vspace{0pt}}
\end{center}

\begin{quotation}
  \pkg{zref-vario} offers a compatibility layer for \pkg{varioref} to be used
  alongside \pkg{zref-clever}.  It provides
  ``\texttt{\textbackslash{}z}\dots{}'' counterparts to \pkg{varioref}'s main
  reference commands, each of which essentially does some (scoped) setup for
  \pkg{varioref}, then calls the original one.
\end{quotation}

\bigskip{}

\begin{center}
  \textbf{EXPERIMENTAL}
\end{center}

\tableofcontents

\clearpage{}

\section{Loading the package}

As usual:

\begin{zvexample}
\usepackage{zref-vario}
\end{zvexample}

\pkg{zref-vario} loads \pkg{zref-clever}, and will load \pkg{varioref} at
\texttt{begindocument/before}, if it hasn't already been loaded.  However,
considering \pkg{varioref} has known load order interaction with other
packages, prominently with \pkg{hyperref}, you may wish to load it explicitly
to ensure its load order requirements are met.


\section{Package options}

Package options can be set by means of \cs{zvsetup}.

\begin{function}{\zvsetup}
  \begin{syntax}
    \cs{zvsetpup}\marg{options}
  \end{syntax}
\end{function}

Available options are:

\bigskip{}

\DescribeOption{pageprop} %
\pkg{varioref} uses the \texttt{page} field of the standard label to check
whether a reference and its label are on the same page, nearby, or far away.
And has no real alternative data to use for the purpose.  Possibly for this
reason, the basic mechanism of page comparison in \pkg{varioref} is restricted
to arabic numbered pages.  Any other page numbering scheme results in
\pkg{varioref} making no distinction between one or many pages off.  But with
\pkg{zref} it is trivial to detach the printed representation of the reference
and the data used to make page comparisons.  And that's what the
\opt{pageprop} option allows to control.  It receives a \pkg{zref} property as
value to be used by \pkg{zref-vario} to make page comparisons, and defaults to
the \texttt{page} property, which corresponds to the usual \pkg{varioref}
behavior.  The property must be defined and be included in \pkg{zref}'s main
list at the time the setting is performed.  The natural intended use case of
this option is the \texttt{abspage} property provided by the
\pkg{zref-abspage} module, which already handles these two requirements.
Thus, loading this module and setting \opt{pageprop} to \texttt{abspage} gives
us page comparisons independent of the page numbering scheme, which works in
the front matter with roman numbering, across front and main matter
boundaries, etc.  In sum, regardless of whether the page numbering is arabic
or anything else.


\section{Reference commands}

\begin{function}{\zvref}
  \begin{syntax}
    \cs{zvref}\meta{*}\oarg{options}\marg{label}
  \end{syntax}
\end{function}

\begin{function}{\zvpageref}
  \begin{syntax}
    \cs{zvpageref}\meta{*}\oarg{options}\marg{label}
  \end{syntax}
\end{function}

\begin{function}{\zvrefrange}
  \begin{syntax}
    \cs{zvrefrange}\meta{*}\oarg{options}\marg{label}\marg{label}
  \end{syntax}
\end{function}

\begin{function}{\zvpagerefrange}
  \begin{syntax}
    \cs{zvpagerefrange}\meta{*}\oarg{options}\marg{label}\marg{label}
  \end{syntax}
\end{function}

\begin{function}{\zfullref}
  \begin{syntax}
    \cs{zfullref}\meta{*}\oarg{options}\marg{label}
  \end{syntax}
\end{function}

These user commands work pretty much as their \pkg{varioref}
counterparts.\footnote{If you are not acquainted with them, see
  \pkg{varioref}'s documentation.}  Indeed, they are just wrappers around
them.  As such, differently from \pkg{zref-clever}'s commands, these can
receive single labels as arguments, not lists of them.  In all of them the
starred version inhibits hyperlinking, and they (locally) set \pkg{varioref}'s
\opt{nospace} option, so that the this syntax can be used unambiguously.

All of them have a single optional argument, which can receive any of
\cs{zcref}'s options, and those are set for calls to \cs{zcref} or
\cs{zpageref} which are part of the building blocks of the reference formats
and strings set for \pkg{varioref}.  However, there is indeed some potential
friction and caveats stemming from the use of these options, which were
designed to work for single calls of \cs{zcref}, to apply to \emph{pairs} of
them.  Hence, whether each and every of these options is meaningful, useful,
or potentially harmful in this context depends on the case, and it is up to
the user to make proper use of them.  Of course, one can always split the pair
using \cs{zcref} and then \cs{zvpageref}, \cs{zvpagerefrange}, or
\cs{zcpageref}, and have more control over each part.  Anyway, the package
does what it says: passes the options along to the underlying call(s) to
\cs{zcref}.\footnote{Note that the options given to each call of these user
  macros are set \emph{before} the call to the original \pkg{varioref}
  counterpart.  This means that options given to \cs{zcref} or \cs{zpageref}
  in the strings and formats in \cs{zvLanguageSetup} will have precedence over
  the former.  Which is useful, so that we can distinguish to some degree the
  first from the second call of \cs{zcref}/\cs{zpageref} done in the formats,
  but does not eliminate the limitations which arise from the underlying
  problem.}

\DescribeOption{vcurrent} %
\DescribeOption{vother} %
Besides \cs{zcref}'s options, mentioned above, two other ones are provided for
\pkg{zref-vario}'s reference commands, corresponding to \pkg{varioref}'s
commands optional arguments: \opt{vcurrent} (the first optional argument) and
\opt{vother} (the second optional argument).  Of course, these are only
available when the underlying \pkg{varioref} command supports them.


\section{Reference format}

\begin{function}{\zvLanguageSetup}
  \begin{syntax}
    \cs{zvLanguageSetup}\marg{language}\marg{options}
  \end{syntax}
\end{function}

User interface for customization of ``strings'' and ``formats'' for
\meta{language}.  \meta{language} must be known to \pkg{zref-clever}.  The
\meta{options} are familiar to users of \pkg{varioref}:
\opt{reftextfaceafter}, \opt{reftextfacebefore}, \opt{reftextafter},
\opt{reftextbefore}, \opt{reftextcurrent}, \opt{reftextfaraway},
\opt{reftextpagerange}, \opt{reftextlabelrange}, \opt{vrefformat},
\opt{vrefrangeformat}, and \opt{fullrefformat}.\footnote{For details, see
  \pkg{varioref}'s documentation.}  Their meaning is the same as the ones they
have in \pkg{varioref} and, indeed, they work by setting those \pkg{varioref}
macros to the values given to the corresponding options in
\cs{zvLanguageSetup}.  If you are setting up a language which has no built-in
support, you should set at least the whole set of ``\texttt{reftext}\dots{}''
options.  Language independent default values are provided for the
``\dots{}\texttt{format}'' options (equivalent to the ones from
\pkg{varioref}, which are adequate for most use cases), so you may omit them.
But, if you need to adjust them, the default values can be overridden by
setting the corresponding options in \cs{zvLanguageSetup}.

As far as \pkg{zref-vario} is concerned, \pkg{varioref}'s options are (mostly)
not taken into account: the language options are disregarded (settings made
with \cs{zvLanguageSetup} are used instead), the \opt{nospace} option is
hard-coded (locally) for the ``\texttt{\textbackslash{}z}\dots{}'' commands,
and \opt{draft} and \opt{final} are typically given to \cs{documentclass},
though they do affect \pkg{zref-vario}'s commands, just as they do with
\pkg{varioref}'s.


\section{Integration with \pkg{zref-check}}

When package \pkg{zref-check} is loaded, \pkg{zref-vario} provides one further
option to its user commands: \opt{vcheck}.  The purpose of this option is to
check the relative position of label and reference within the same page.  It
can receive two values: \texttt{above} and \texttt{below}, being those the
names of the \pkg{zref-check}'s checks which are meaningful for the use case
at hand.  Of course, these could also be performed with \cs{zcref}'s
\opt{check} option, which is available for \pkg{zref-vario}'s commands as
well.  The difference here is that the check specified in \opt{vcheck} is only
performed when the referenced label -- or labels, in the case of a range --
fall on the same page as the reference itself.\footnote{There's another
  technical difference between them.  \pkg{zref-check}'s checks, and hence
  those of the \opt{check} option, make sure the whole reference passes each
  check by setting labels both at the start and the end of the reference, and
  verifying if each one of them passes the checks.  But, since \pkg{varioref}
  already has it's own mechanism to handle references which cross page
  boundaries, \opt{vcheck}'s checks set only one label, at the end of the
  reference, the same position \pkg{varioref} uses to check whether label and
  reference are on the same page.}  In other words, when \opt{vcurrent} would
be used (if provided).  For this reason, only the commands which support
\opt{vcurrent} offer \pkg{vcheck}.  Also, \opt{vcheck} cannot receive
\pkg{zref-check}'s options, as \opt{check} is able to, but the latter can be
used for the purpose of locally controlling the behavior of the checks
performed by \opt{vcheck}.  Consult \pkg{zref-check}'s documentation for
details and limitations of these checks and envisaged workflows for their
reliable use.


% \section{Acknowledgments}
%
% A number of people have contributed to \pkg{zref-vario}.  Suggestions,
% ideas, solutions to problems, bug reports or even encouragement were
% generously provided by (in chronological order):
%
% The package's language settings have been provided or improved thanks to:
%
% If I have inadvertently left anyone off the list I apologize, and please let
% me know, so that I can correct the oversight.
%
% Thank you all very much!


\section{Change history}

A change log with relevant changes for each version, eventual upgrade
instructions, and upcoming changes, is maintained in the package's repository,
at \url{https://github.com/gusbrs/zref-vario/blob/main/CHANGELOG.md}.  The
change log is also distributed with the package's documentation through CTAN
upon release so, most likely, \texttt{texdoc zref-vario/changelog} should
provide easy local access to it.  An archive of historical versions of the
package is also kept at \url{https://github.com/gusbrs/zref-vario/releases}.

\end{document}
