% \iffalse meta-comment
%
% File: zref-vario.tex
%
% This file is part of the LaTeX package "zref-vario".
%
% Copyright (C) 2021-2022  Gustavo Barros
%
% It may be distributed and/or modified under the conditions of the
% LaTeX Project Public License (LPPL), either version 1.3c of this
% license or (at your option) any later version.  The latest version
% of this license is in the file:
%
%    https://www.latex-project.org/lppl.txt
%
% and version 1.3 or later is part of all distributions of LaTeX
% version 2005/12/01 or later.
%
%
% This work is "maintained" (as per LPPL maintenance status) by
% Gustavo Barros.
%
% This work consists of the files zref-vario.dtx,
%                                 zref-vario.ins,
%                                 zref-vario.tex,
%                                 zref-vario-code.tex,
%         and the files listed in MANIFEST.md.
%
% The released version of this package is available from CTAN.
%
% -----------------------------------------------------------------------
%
% The development version of the package can be found at
%
%    https://github.com/gusbrs/zref-vario
%
% for those people who are interested.
%
% -----------------------------------------------------------------------
%
% \fi

\documentclass{l3doc}

% The package itself *must* be loaded so that \GetFileInfo can pick up date
% and version data.
\usepackage{zref-vario}

\usepackage[T1]{fontenc}

\usepackage[sc]{mathpazo}
\linespread{1.05}
\usepackage[scale=.88]{tgheros} % sans
\usepackage[varqu,scaled=1.03]{inconsolata} % tt

\usepackage{listings}

\usepackage{microtype}

\hypersetup{hidelinks}

\NewDocumentCommand\opt{m}{\texttt{#1}}

\lstdefinestyle{code}{
  language=[LaTeX]TeX,
  % moretexcs={
  % }
}
\lstdefinestyle{zrefvario}{
  style=code,
  % moretexcs={
  % }
}
\lstset{
  style=zrefvario,
  basicstyle=\ttfamily\small,
  columns=fullflexible,
  keepspaces,
  xleftmargin=\leftmargin,
  xrightmargin=.5\leftmargin,
}
\lstnewenvironment{zvexample}[1][]{%
  \renewcommand{\lstlistingname}{Example}%
  \lstset{#1}%
}{}

\begin{document}

\GetFileInfo{zref-vario.sty}

\title{%
  The \pkg{zref-vario} package%
  \thanks{This file describes \fileversion, released \filedate.}%
  \texorpdfstring{\\{}\medskip{}}{ - }%
  User manual%
  \texorpdfstring{\medskip{}}{}%
}

\author{%
  Gustavo Barros%
  \thanks{\url{https://github.com/gusbrs/zref-vario}}%
}

\date{\filedate}

\maketitle

\begin{center}
  {\bfseries \abstractname\vspace{-.5em}\vspace{0pt}}
\end{center}

\begin{quotation}
  \pkg{zref-vario} offers a compatibility layer for \pkg{varioref} to be used
  alongside \pkg{zref-clever}.  It provides ``\texttt{\cs{z}}\dots{}''
  counterparts to \pkg{varioref}'s main reference commands, each of which
  essentially does some (scoped) setup for \pkg{varioref}, then calls the
  original one.
\end{quotation}

\bigskip{}

\begin{center}
  \textbf{EXPERIMENTAL}
\end{center}

\tableofcontents

\clearpage{}

\section{Loading the package}

As usual:

\begin{zvexample}
\usepackage{zref-vario}
\end{zvexample}


\section{Dependencies}

\pkg{zref-vario} requires \pkg{zref-clever} and \pkg{varioref}.


\section{User interface}

\begin{function}{\zvref}
  \begin{syntax}
    \cs{zvref}\meta{*}\oarg{options}\marg{label}
  \end{syntax}
\end{function}

\begin{function}{\zvpageref}
  \begin{syntax}
    \cs{zvpageref}\meta{*}\oarg{options}\marg{label}
  \end{syntax}
\end{function}

\begin{function}{\zvrefrange}
  \begin{syntax}
    \cs{zvrefrange}\meta{*}\oarg{options}\marg{label}\marg{label}
  \end{syntax}
\end{function}

\begin{function}{\zvpagerefrange}
  \begin{syntax}
    \cs{zvpagerefrange}\meta{*}\oarg{options}\marg{label}\marg{label}
  \end{syntax}
\end{function}

\begin{function}{\zfullref}
  \begin{syntax}
    \cs{zfullref}\meta{*}\oarg{options}\marg{label}
  \end{syntax}
\end{function}

These user commands work pretty much as their \pkg{varioref} counterparts.
Indeed, they are just wrappers around them.  As such, differently from
\pkg{zref-clever}'s commands, these can receive single labels as arguments,
not lists of them.  In all of them the starred version inhibits hyperlinking,
and they (locally) set \pkg{varioref}'s \opt{nospace} option, so that the
this syntax can be used unambiguously.

All of them have a single optional argument, which can receive any of
\cs{zcref}'s options, and those are set for calls to \cs{zcref} or
\cs{zpageref} which are part of the building blocks of the reference formats
and strings set for \pkg{varioref}.  However, there is indeed some potential
friction and caveats stemming from the use of these options, which were
designed to work for single calls of \cs{zcref}, to apply to \emph{pairs} of
them.  Hence, whether each and every of these options is meaningful, useful,
or potentially harmful in this context depends on the case, and it is up to
the user to make proper use of them.  Of course, one can always split the pair
using \cs{zcref} and then \cs{zvpageref}, \cs{zvpagerefrange}, or
\cs{zcpageref}, and have more control over each part.  Anyway, the package
does what it says: passes the options along to the underlying call(s) to
\cs{zcref}.\footnote{Note that the options given to each call of these user
  macros are set \emph{before} the call to the original \pkg{varioref}
  counterpart.  This means that options given to \cs{zcref} or \cs{zpageref}
  in the strings and formats in \cs{zvLanguageSetup} will have precedence over
  the former.  Which is useful, so that we can distinguish to some degree the
  first from the second call of \cs{zcref}/\cs{zpageref} done in the formats,
  but does not eliminate the limitations which arise from the underlying
  problem.}

Besides these options, two other are provided corresponding to
\pkg{varioref}'s commands' optional arguments: \opt{vcurrent} (the first
optional argument) and \opt{vother} (the second optional argument).  Of
course, these are only available when the underlying \pkg{varioref} command
supports them.

\section{Customization}


\begin{function}{\zvLanguageSetup}
  \begin{syntax}
    \cs{zvLanguageSetup}\marg{language}\marg{options}
  \end{syntax}
\end{function}

User interface for customization of ``strings'' and ``formats'' for
\meta{language}.  \meta{language} must be known to \pkg{zref-clever}.  The
\meta{options} are familiar to users of \pkg{varioref}:
\opt{reftextfaceafter}, \opt{reftextfacebefore}, \opt{reftextafter},
\opt{reftextbefore}, \opt{reftextcurrent}, \opt{reftextfaraway},
\opt{reftextpagerange}, \opt{reftextlabelrange}, \opt{vrefformat},
\opt{vrefrangeformat}, and \opt{fullrefformat}.\footnote{For details, see
  \pkg{varioref}'s documentation.}  Their meaning is the same as the ones they
have in \pkg{varioref} and, indeed, they work by setting those \pkg{varioref}
macros to the values given to the corresponding options in
\cs{zvLanguageSetup}.  If you are setting up a language which has no built-in
support, you should set at least the whole set of ``\texttt{reftext}\dots{}''
options.  Language independent default values are provided for the
``\dots{}\texttt{format}'' options (equivalent to the ones from
\pkg{varioref}, which are adequate to most use cases), so you may omit them.
But, if you need to adjust them, the default values can be overridden by
setting the corresponding options in \cs{zvLanguageSetup}.


% \section{Acknowledgments}
%
% A number of people have contributed to \pkg{zref-vario}.  Suggestions,
% ideas, solutions to problems, bug reports or even encouragement were
% generously provided by (in chronological order):
%
% The package's language settings have been provided or improved thanks to:
%
% If I have inadvertently left anyone off the list I apologize, and please let
% me know, so that I can correct the oversight.
%
% Thank you all very much!


\section{Change history}

A change log with relevant changes for each version, eventual upgrade
instructions, and upcoming changes, is maintained in the package's repository,
at \url{https://github.com/gusbrs/zref-vario/blob/main/CHANGELOG.md}.  An
archive of historical versions of the package is also kept at
\url{https://github.com/gusbrs/zref-vario/releases}.

\end{document}
